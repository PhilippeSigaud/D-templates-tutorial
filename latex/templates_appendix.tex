\newpage
\part*{Appendices}
\addcontentsline{toc}{part}{Appendices}
\appendix

\section{
\texorpdfstring{A Crash Course on the \D{is}\DD{(...)} Expression}
               {A Crash Course on the is(...) expression}}
\label{isexpression}

\subsection{General Syntax} \label{issyntax}

The \D{is}\DD{(...)} expression gives you some compile-time introspection\index{compile-time!introspection} on types (and, as a side-effect, on D expressions). It's described \href{http://www.dlang.org/expression.html#IsExpression}{here} in the D website. This expression has a quirky syntax, but the basic use is very simple and it's quite useful in conjunction with \D{static if} (see section \ref{staticif}) and template constraints (see section \ref{constraints}). The common syntaxes are:

\index{syntax!is expression@\D{is} expression}
\begin{dcode}
is( Type (optional identifier) )
is( Type (optional identifier) :  OtherType, 
         (optional template parameters list) )
is( Type (optional identifier) == OtherType, 
         (optional template parameters list) )
\end{dcode}

If what's inside the parenthesis is valid (see below), \D{is}\DD{()} returns \D{true} at compile-time, else it returns \D{false}.

\subsection{\texorpdfstring{\D{is}\DD{(Type)}}
                           {is(Type)}}
\label{istype}

Let's begin with the very first syntax: if \DD{Type} is a valid D type in the scope\index{scope!local scope} of the \D{is} expression, \D{is}\DD{()} returns \D{true}. As a bonus, inside a \D{static if}, the optional \DD{identifier} becomes an alias for the type. For example:

\index{is expression@\D{is} expression}
\begin{dcode}
module canbeinstantiated;

template CanBeInstantiatedWith(alias templateName, Types...)
{
    // is templateName!(Types) a valid type?
    static if (is( templateName!(Types) ResultType ))
    // here you can use ResultType (== templateName!(Types))
        alias ResultType CanBeInstantiatedWith;
    else
        alias void       CanBeInstantiatedWith;
}
\end{dcode}

Note that the previous code was done with templates in mind, but it is quite robust: if you pass as \DD{templateName} something that's not a template name (a func\-tion name, for ex\-am\-ple), the \D{is} will see \DD{templateName!(Types)} has no valid type and will return \D{false}.  \DD{CanBeInstantiatedWith} will cor\-rec\-tly be set to \D{void} and your program does not crash.

\aparte{Testing for an alias}{Sometimes you do not know if the template argument you received is a type or an alias (for example, when dealing with tuples elements). In that case, you can use \DD{!}\D{is}\DD{(symbol)} as a test. If it really is an alias and not a type, this will return \D{true}.}

An interesting use for this form of \D{is}, is testing whether or not some D code is valid. Consider: D blocks are seen as delegates by the compiler (Their type is \D{void delegate}\DD{()}). Using this in conjunction with \D{typeof} let you test the validity of a block statement: if \DD{some code} is valid, \D{typeof}\DD{(\{ some code \}())} (note the \DD{()} at the end of the delegate to 'activate' it) has a real D type and \D{is} will return true.

Let's put this to some use. Imagine you have a function template \DD{fun} and some arguments, but you do not know if \DD{fun} can be called with this particular bunch of arguments. If it's a common case in your code, you should abstract it as a template. Let's call it \DD{validCall} and make it a function template also, to easily use it with the arguments:

\index{is(typeof())@\D{is}\DD{(}\D{typeof}\DD{())}}
\begin{dcode}
module validcall;
import std.conv;

bool validCall(alias fun, Args...)(Args args) 
{
    return is( typeof({ /* code to test */
                        fun(args);
                        /* end of code to test */
                      }()))
}

// Usage:
T add(T)(T a, T b) { return a+b;}
string conc(A,B)(A a, B b) { return to!string(a) ~ to!string(b);}

void main()
{
    assert( validCall!add(1, 2));   // generates add!(int)
    assert(!validCall!add(1, "abc")); // no template instantiation possible

    assert( validCall!conc(1, "abc")); // conc!(int, string) is OK.
    assert(!validCall!conc(1)       ); // no 1-argument version for conc

    struct S {}
  
    assert(!validCall!S(1, 2.3)); // S is not callable
}
\end{dcode}

Note that the tested code is simply `\DD{fun(args);}'. That is, there is no condition on \DD{fun}'s type: it could be a function, a delegate or even a struct or class with \D{opCall} defined. There are basically two ways \DD{fun(args);} can be invalid code: either \DD{fun} is not callable as a function, or it is callable, but \DD{args} are not valid arguments.

By the way, fun as it may be to use this trick, D provides you with a cleaner way to test for valid compilation:

\begin{dcode}
__traits(compiles, { /* some code */ })
\end{dcode}

\D{\_\_traits} is another of D numerous Swiss Army knife constructs. You can find the \DD{compiles} documentation \href{http://www.dlang.org/traits.html#compiles}{here}. Section \ref{traits} is dedicated to it.

\subsection{
\texorpdfstring{\D{is}\DD{(Type : AnotherType)} and \D{is}\DD{(Type == AnotherType)}}
               {is(Type : AnotherType) and is(Type == AnotherType)}}
\label{istypeanothertype}

The two other basic forms of \D{is} return true if \DD{Type} can be implicitly converted to (is derived from) \DD{AnotherType} and if \DD{Type} is exactly \DD{AnotherType}, respectively.  I find them most interesting in their more complex form, with a list of template parameters afterwards. In this case, the template parameters act a bit as type variables in an equation. Let me explain:

\index{syntax!is expression@\D{is} expression}
\begin{dcode}
is(Type identifier == SomeComplexTypeDependingOnUAndV, U, V)
\end{dcode}

really means: `excuse me, Mr. D Compiler, but is \DD{Type} perchance some complex type depending on \DD{U} and \DD{V} for some \DD{U} and \DD{V}? If yes, please give me those.'
For 

\index{is expression@\D{is} expression}
\begin{dcode}
template ArrayElement(T)
{
    // is T an array of U, for some U?
    static if (is(T t : U[], U)) 
        alias U ArrayElement; // U can be used, let's expose it
    else
        alias void ArrayElement;
}

template isAssociativeArray(AA)
{
    static if (is( AA aa == Value[Key], Value, Key))  
        /* code here can use Value and Key, 
           they have been deduced by the compiler. */ 
    else
        /* AA is not an associative array
          Value and Key are not defined. */
}
\end{dcode}

Strangely, you can only use it with the \D{is}\DD{(Type identifier, ...)} syntax: you \emph{must} have \DD{identifier}. The good new is, the complex types being inspected can be templated types and the parameter list can be any template parameter: not only types, but integral values, \ldots.
For example, suppose you do what everybody does when encountering D templates: you create a templated  n-dimensional vector type.

\begin{dcode}
struct Vector(Type, int dim) { ... }
\end{dcode}

If you did not expose \DD{Type} and \DD{dim} (as aliases for example, as seen in sections \ref{inneralias} and \ref{givingaccess}), you can use \D{is} to extract them for you:

\begin{dcode}
Vector!( ?, ?) myVec;
// is myVec a vector of ints, of any dimension?
static if (is(typeof(myVec) mv == Vector!(int, dim), dim))

// is it a 1-dimensional vector?
static if (is(typeof(myVec) mv == Vector!(T, 1), T))
\end{dcode}

\aparte{is( A != B)?}{No, sorry, this doesn't exist. Use \DD{!}\D{is}\DD{(A == B)}. But beware this will also fire if \DD{A} or \DD{B} are not legal types (which makes sense: if \DD{A} is not defined, then by definition it cannot be equal to \DD{B}). If necessary, you can use \D{is}\DD{(A) \&\& }\D{is}\DD{(B) \&\& !}\D{is}\DD{(A == B)}.}

\aparte{is A a supertype to B?}{ Hey, \D{is}\DD{(MyType : SuperType)} is good to know if \DD{MyType} is a subtype to \DD{SuperType}. How do I ask if \DD{MyType} is a supertype to \DD{SubType}? Easy, just use \D{is}\DD{(SubType : MyType)}.}

For me, the main limitation\index{is expression@\D{is} expression!limitation} is that template tuple parameters are not accepted. Too bad. See, imagine you use \stdanchor{typecons}{Tuple} a lot. At one point, you need a template to test if something is a \DD{Tuple!(T...)} for some \DD{T} which can be 0 or more types. Though luck, \D{is} is a bit of a letdown there, as you cannot do:

\begin{dcode}
template isTuple(T)
{
    static if (is(T tup == Tuple!(InnerTypes), InnerTypes...)
(...)                                          ^^^^^^^^^^^^^
\end{dcode}

But sometimes D channels its inner perl\index{perl!inner perl in D} and, lo! There is more than one way to do it! You can use IFTI\index{IFTI} (see \ref{ifti}) and our good friend the \D{is}\DD{(}\D{typeof}\DD{({...}()))} expression there. You can also use \D{\_\_traits}, depending on you mood, but since this appendix is specifically on \D{is}:

\index{is(typeof())@\D{is}\DD{(}\D{typeof}\DD{())}}
\begin{ndcode}
module istuple;
import std.typecons;

template isTuple(T)
{
    static if (is(typeof({
              void tupleTester(InnerTypes...)(Tuple!(InnerTypes) tup) {}
              T.init possibleTuple;
              tupleTester(possibleTuple);
              }())))
        enum bool isTuple = true;
    else
        enum bool isTuple = false;
}
\end{ndcode}

Line 7 defines the function template \DD{tupleTester}, that only accepts \DD{Tuple}s as arguments (even though it does nothing with them). We create something of type \DD{T} on line 8, using the \DD{.init} property inherent in all D types, and try to call \DD{tupleTester} with it. If \DD{T} is indeed a \DD{Tuple} this entire block statement is valid, the resulting delegate call indeed has a type and \D{is} returns \D{true}.

There are two things to note here: first, \DD{isTuple} works for any templated type called \DD{Tuple}, not only \stdanchor{typecons}{Tuple}. If you want to restrict it, change \DD{tupleTester} definition. Secondly, we do not get access to the inner types this way. For \stdanchor{typecons}{Tuple} it's not really a problem, as they can be accessed with the \DD{someTuple.Types} alias, but still\ldots

By the way, the template parameter list elements can themselves use the \DD{A : B} or \DD{A == B} syntax:

\begin{dcode}
static if (is( T t == A!(U,V), U : SomeClass!W, V == int[n], W, int n))
\end{dcode}

This will be OK if \DD{T} is indeed an \DD{A} instantiated with an \DD{U} and a \DD{V}, themselves verifying that this \DD{U} is derived from \DD{SomeClass!W} for some \DD{W} type and that \DD{V} is a static array of \D{int}s of length \DD{n} to be determined (and possibly used afterwards). In the if branch of the \D{static if} \DD{U}, \DD{V}, \DD{W} and \DD{n} are all defined. 

\subsection{Type Specializations}\label{typespecializations}

There is a last thing to know about \D{is}: with the \D{is}\DD{(Type (identifier) == Something)} version, \DD{Something} can also be a type specialization\index{is expression@\D{is} expression!type specialization}, one of the following D keywords: \D{function}, \D{delegate}, \D{return}, \D{struct}, \D{enum}, \D{union}, \D{class}, \D{interface}, \D{super}, \D{const}, \D{immutable} or \D{shared}. The condition is satisfied if \DD{Type} is one of those (except for \D{super} and \D{return}, see below). \DD{identifier} then becomes an alias for some property of \DD{Type}, as described in table \ref{table:typespecializations}.

\begin{table}[htb]
\begin{adjustwidth}{-0.2cm}{} % long table: reducing left margin
\begin{tabular}[c]{cp{9em}p{17em}}
\hline
\multicolumn{1}{c}{Specialization} & \multicolumn{1}{c}{Satisfied if} &  \multicolumn{1}{c}{\DD{identifier} becomes} \\ \hline% \hline
\D{function}& \DD{Type} is a function & The function parameters type tuple \\ %\hline
\D{delegate}& \DD{Type} is a delegate & The delegate function type \\ %\hline
\D{return}& \DD{Type} is a function  & The return type \\ 
          & \DD{Type} is a delegate & \\ \hline %\hline
\D{struct}& \DD{Type} is a struct & The struct type\\ %\hline
\D{enum}& \DD{Type} is an enum & The enum base type\\ %\hline
\D{union}& \DD{Type} is an union& The union type\\ %\hline
\D{class}& \DD{Type} is a class& The class type\\ %\hline
\D{interface}& \DD{Type} is an interface & The interface type\\ %\hline
\D{super}& \DD{Type} is a class& The type tuple (Base Class, Inter\-fa\-ces)\\ \hline %\hline
\D{const}& \DD{Type} is const & The type\\ %\hline
\D{immutable}& \DD{Type} is immutable & The type\\ %\hline
\D{shared}& \DD{Type} is shared & The type\\ \hline
\end{tabular}
\caption{Effect of type specializations in \D{is}}
\label{table:typespecializations}
\end{adjustwidth}
\end{table}

Let's put that to some use: we want a factory template that will create a new struct or a new class, given its name as a template parameter:

\index{is expression@\D{is} expression!type specialization}
\begin{dcode}
module maker;
import std.algorithm;
import std.conv;

template make(A) 
    if (is(A a == class ) 
     || is(A a == struct))
{
    auto make(Args...)(Args args)
    {
        static if (is(A a == class))
            return new A(args);
        else
            return A(args);
    }
}

struct S {int i;}
class C 
{
    int i; 
    this(int ii) { i = ii;}
    override string toString() @property { return "C("~to!string(i)~")";}
}

void main()
{
    auto array = [0,1,2,3];

    auto structRange = map!( make!S )(array);
    auto classRange  = map!( make!C )(array);

	 assert(equal(structRange, [S(0), S(1), S(2), S(3)]));
    assert(classRange.front.toString == "C(0)");
}
\end{dcode}

You can find another example of this kind of \D{is} in section \ref{classtemplates}, with the \DD{duplicator} template.

\section{Resources and Further Reading}
\label{resources}

\unfinished{Any new resource welcome!}

Being a relatively young language, D doesn't have dozens of books or online sites dedicated to its learning. Still, there are a few resources on the Internet that pertains to D templates and associated subjects.

\subsection{D Templates}

Here are some resources on D templates.

\subsubsection{D Reference}
\label{dreference}

Your first stop should be the \href{http://dlang.org}{dlang.org} pages on templates. They are quite easy to read, don't hesitate to peruse them and bother the authors if anything is unclear or blatantly false. The \DD{dlang.org} website is also a github project, \href{https://github.com/D-Programming-Language/d-programming-language.org}{here}. Table \ref{table:dlangpages} lists the pages dealing with templates.

 
\begin{table}[htb]
\begin{adjustwidth}{-1.3cm}{} % long table: reducing left margin
\begin{tabular}[l]{p{14em}l}
\hline
\multicolumn{1}{c}{Subject} & \multicolumn{1}{c}{URL}\\ \hline %\hline
The general page on templates & \url{http://dlang.org/template.html}\\ %\hline
D templates vs C++ ones & \url{http://dlang.org/templates-revisited.html} \\ %\hline
An article on tuples & \url{http://dlang.org/tuple.html}\\ %\hline
Template tuple parameters & \url{http://dlang.org/variadic-function-templates.html} \\ %\hline
Template constraints & \url{http://dlang.org/concepts.html}\\ %\hline
Mixin Templates & \url{http://dlang.org/template-mixin.html}\\ %\hline
A small article on string mixins & \url{http://dlang.org/mixin.html}\\% \hline
\D{static if} & \url{http://dlang.org/version.html#staticif}\\ %\hline
\D{is} & \url{http://dlang.org/expression.html#IsExpression}\\ %\hline
Type Aliasing & \url{http://dlang.org/declaration.html#alias}\\ %\hline
Operator Overloading & \url{http://dlang.org/operatoroverloading.html}\\ \hline
\end{tabular}
\caption{\url{dlang.org} pages dealing with templates}
\label{table:dlangpages}
\end{adjustwidth}
\end{table}

\subsubsection{The D Programming Language}
\label{TDPL}

The main D book as of this writing is of course \textsc{The D Programming Language} (also known as \textsc{TDPL}), by Andrei Alexandrescu. It's a very good reference book, and fun to read too. TDPL has a very interesting approach to templates: Andrei first introduces them as a natural extension of functions, classes, and structs syntax, to get parametrized code. It's only afterwards that templates-as-scopes are introduced. Table \ref{table:tdplchapters} lists the chapters that deals with templates. 

\begin{table}[htb]
\centering
\begin{tabular}[c]{ll}
\hline
\multicolumn{1}{c}{Subject} & \multicolumn{1}{c}{Chapter} \\ \hline
Functions Templates & 5.3, 5.4, 5.10, and 5.12\\
Classes Templates & 6.14 \\
Struct Templates & 7.1.10 \\ 
\D{alias} & 7.4 \\ 
General Templates & 7.5 \\ 
Mixin Templates & 7.6 \\ 
Operator Overlaoding & 12 \\ \hline
\end{tabular}
\caption{TDPL Chapters Dealing With Templates}
\label{table:tdplchapters}
\end{table}

\subsubsection{Programming in D}
\label{programmingind}

Another quite interesting book is the translation from Turkish of Ali \c{C}ehreli's book \textsc{Programming in D}, which you can find for free here: \url{http://ddili.org/ders/d.en/index.html}. More precisely, as far as this document is concerned, the chapter on templates is already translated and can be found at \url{http://ddili.org/ders/d.en/templates.html}. Ali's book aims to be a gentle introduction to D, so the rhythm is less gung ho than TDPL. If you do not want to read the present document,\footnote{ It's probably too late. Should I link to Ali's book earlier?} the previously linked chapter is your best bet as a template tutorial.

\subsubsection{The D Wiki}
\label{dwiki}

Apart from books, the D wiki has a tutorial on templates that can be found at \url{http://prowiki.org/wiki4d/wiki.cgi?D__Tutorial/D2Templates}, that explains \stdanchor{functional}{unaryFun} very thoroughly, dealing in passing with various subjects such as templates, string mixins, and instantiation scope. I found it a very pleasant read and a nice explanation of a difficult part of Phobos.

\subsection{D Metaprogramming}
\label{dmetaprogramming}

I found Nick Sabalausky's article at \url{http://www.semitwist.com/articles/EfficientAndFlexible/SinglePage/} to be a very good read, and full of nifty ideas. In particular, Nick shows how to use compile-time information to craft code such that runtime arguments can then be used to ``influence'' compile-time values. Yes you read that right, and my description does not do it justice: \href{http://www.semitwist.com/articles/EfficientAndFlexible/SinglePage/#part6-3}{read on}, and see for yourselves. I definitely \emph{must} put a section somewhere in this doc on this.

The code shown in the article can be found in a github project, at \url{https://github.com/Abscissa/efficientAndFlexible}

\subsection{Templates in Other Languages}
\label{templatesinotherlanguages}

\unfinished{C++ templates, obviously. Articles, online books, and such. A small text on Boost. Also, maybe Java and C\# generics.}

\subsection{Genericity in Other Languages}
\label{genericity}

\unfinished{Haskell and ML concerning types. In general, any type trick that Haskellers live by: encoding information in types, using types to forbid unwanted states and so on.}

\subsection{Metaprogramming in Other Languages}
\subsection{macros}

\unfinished{Maybe Lisp / Scheme / Clojure macros. Nemerle macros too.}

